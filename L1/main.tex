\documentclass{article}
\usepackage{subcaption}
\usepackage{amsmath}
\usepackage{mathtools}
\usepackage{hyperref}
\usepackage{polski}
\usepackage[utf8]{inputenc}
\usepackage{titlesec}
\usepackage{scalerel}
\usepackage{upquote}
\usepackage{multicol}
\usepackage{graphicx}
\usepackage{tabularx}
\usepackage{blindtext}
\usepackage{color}
\usepackage{colortbl}
\usepackage{amsmath}
\usepackage{algorithm}
\usepackage[noend]{algpseudocode}
\usepackage{multicol}
\usepackage{braket}
\usepackage[dvipdfmx]{media9}
\definecolor{dkgreen}{rgb}{0,0.6,0}
\definecolor{gray}{rgb}{0.5,0.5,0.5}
\definecolor{mauve}{rgb}{0.58,0,0.82}
\usepackage{geometry}
 \geometry{
 a4paper,
 total={170mm,257mm},
 left=20mm,
 top=20mm,
 }
\usepackage{placeins}
\usepackage{svg}
\newcolumntype{s}{>{\columncolor{yellow}}c}
\newcolumntype{h}{>{\columncolor{pink}}c}
\usepackage{listings}
\usepackage{float}

\lstset{frame=tb,
  language=Matlab,
  aboveskip=3mm,
  belowskip=3mm,
  showstringspaces=false,
  columns=flexible,
  basicstyle={\small\ttfamily},
  numbers=none,
  numberstyle=\tiny\color{gray},
  keywordstyle=\color{blue},
  commentstyle=\color{dkgreen},
  stringstyle=\color{mauve},
  breaklines=true,
  breakatwhitespace=true,
  tabsize=3
}
\usepackage[utf8]{inputenc}
\titleclass{\subsubsubsection}{straight}[\subsection]

\newcounter{subsubsubsection}[subsubsection]
\renewcommand\thesubsubsubsection{\thesubsubsection.\arabic{subsubsubsection}}

\titleformat{\subsubsubsection}
  {\normalfont\normalsize\bfseries}{\thesubsubsubsection}{1em}{}
\titlespacing*{\subsubsubsection}
{0pt}{3.25ex plus 1ex minus .2ex}{1.5ex plus .2ex}
\makeatletter
\def\toclevel@subsubsubsection{4}
\def\l@subsubsubsection{\@dottedtocline{4}{7em}{4em}}
\makeatother
\setcounter{secnumdepth}{4}
\setcounter{tocdepth}{4}


\title{Podstawy Obliczeń Komputerowych\\ Laboratorium 1}
\author{Kamil Czop 259613 \\ Łukasz Majchrzak 262761 \\ Wtorek 11:15TN - Y02-37c}
\date{28 marca 2023}

\begin{document}

\maketitle

\section{Cel ćwiczeń}
Celem przeprowadzonych laboratoriów było utworzenie programu rozwiązującego układ równań liniowych z wykorzystaniem dowolnej metody skończonej.
\section{Dane}
\begin{figure}[H]
    \begin{multicols}{3}
        \begin{equation*}
         \mathbf{A}=\begin{bmatrix}2 & 1 & 1 & -1\\1 & 1 & 1 & -1\\ -1 & 2 & -1 & 1 \\ -1 & 2 & -1 & 1 \\\end{bmatrix}
        \end{equation*}
    \par
        \begin{equation*}
         \mathbf{B}=\begin{bmatrix}3 \\ 4 \\ 10 \\ -4\\\end{bmatrix}
        \end{equation*}
    \par
        \begin{equation*}
         \mathbf{AB}=\begin{bmatrix}2 & 1 & 1 & -1 & 3\\1 & 1 & 1 & -1 & 4\\ -1 & 2 & -1 & 1 & 10\\ -1 & 2 & -1 & 1 & 4\\\end{bmatrix}
        \end{equation*}
    \end{multicols}
\end{figure}

\section{Wykonanie} 
\subsection{Twierdzenie Kroneckera-Capellego}
Bazując na wymaganiach określonych w instrukcji laboratoryjnej wykonano początkowo program określającego z tw. Kroneckera-Capellego do wyznaczenia liczby rozwiązań naszego układu równań.
Po przepuszczeniu przez algorytm obliczający rangę macierzy w C++ i gotowej funkcji Matlab rank, dla macierzy $\mathbf{A}$ uzyskano wartość 4, a dla $
\mathbf{AB}$ także 4. Z tego wynika iż nasz układ równań posiadając równe rangi może posiadać jedno rozwiązanie układu. Posiadając potwierdzenie istnienia rozwiązania możemy rozwiązać układ metodą Gauss'a-Jordana.\\
\subsection{Metoda Gauss'a - Jordana}
\begin{algorithm}[H]
    \caption{Algorytm postępowania dla metody Gauss'a-Jordana}\label{euclid}
    \begin{algorithmic}[1]
        \Procedure{gaussJordan}{}
        \For {$i = 1 to n$}
            \If {$A_{i,i} = 0$}
                \State \textbf{Stop}
            \EndIf
            \For {$j = 1 to n$}
                \If{ $i != j$ }
                    \State $Ratio \gets A_{j,i} / A{i,i}$
                    \For {$k = 1 to n$}
                        \State {$A_{j,k} \gets A_{j,k} - Ratio * A{i,k}$}
                    \State \textbf{Next} $k$
                    \EndFor
                \EndIf
            \State \textbf{Next} $j$
            \EndFor
        \State \textbf{Next} $i$
        \EndFor
        \State \textbf{Stop}
        \EndProcedure
    \end{algorithmic}
\end{algorithm}
Metoda ta stanowi modyfikacje metody Gauss'a gdzie układ  równań liniowych jest sprowadzany do macierzy jednostkowej. W początkowej faczie poszerzamy macierz $\mathbf{A}$ o macierz $\mathbf{B}$ uzyskując macierz $\mathbf{AB}$. Macierz $AB$ przechodzi podaną transformacje. \\
\[
\begin{bmatrix} 
    A_{11} & \dots & A_{1n} & B_{1} \\
    \vdots & \ddots & \vdots & \vdots \\
    a_{n1} &  \dots & A_{nn}     & B_{n} 
    \end{bmatrix}
\qquad
\rightarrow \begin{bmatrix} 
    1 & \dots & 0 & B_{1}^{(n)} \\
    \vdots & \ddots & \vdots & \vdots \\
    0 &  \dots & 1    & B_{n}^{(n)} 
    \end{bmatrix}
\]\\
Wynikiem operacji na wykorzystywanym zbiorze danych jest macierz jednostkowa o postaci:
\begin{center}
    \begin{equation*}
        \mathbf{AB}=\begin{bmatrix}1 & 0 & 0 & 0 & 1\\0 & 1 & 0 & 0 & 2\\ 0 & 0 & 1 & 0 & 3\\ 0 & 0 & 0 & 1 & 4\\\end{bmatrix}
    \end{equation*}
\end{center}
Co przekłada się na uzyskanie współczynników o wartościach:
\begin{figure}[H]
    \begin{multicols}{4}
        \begin{equation*}
            x_1 = 1
        \end{equation*}
        \par
        \begin{equation*}
            x_2 = 2
        \end{equation*}
        \par
        \begin{equation*}
            x_3 = 3
        \end{equation*}
        \par
        \begin{equation*}
            x_4 = 4
        \end{equation*}
    \end{multicols}
\end{figure}

\section{Kod źródłowy}
\subsection{Kod funkcji wykonującej metodę gauss'a-jordana}
\lstinputlisting[language=cxx]{Cpp/gjor.cpp}

\section{Github, bibliografia, dokumentacja}
\begin{itemize}
    \item \url{https://github.com/Myknakryu/pok-2023}
    \item \url{https://traf-barak.pwr.edu.pl/wp-content/uploads/2020/08/Raport.pdf}
    \item \url{https://www.youtube.com/watch?v=xOLJMKGNivU}
\end{itemize}
\end{document}
